\section{Sprint Material} % (fold)
\label{sec:Sprint Material}
\subsection{Version} % (fold)
\label{sub:Version}
The current state of our app has version 0.0.4. Meaning that the app is still alpha quality, with some functionality implementet, but still contains unfinished work.
% subsection Version (end)
\subsection{Source code} % (fold)
\label{sub:Source code}
We still use Git and GitHub. The source code is available at: \url{https://github.com/mundane/ETA_analytics}, in the app folder. The other folders contains prototypes, experiments, and reports.
% subsection Source code (end)
\subsection{Sprint Explanation}% TODO
This was the last sprint in the project, which meant that many tasks had to be finalized in this sprint. Work was therefore focused on this instead of trying to implement new user stories.
In this sprint we prioritized to work together, which can be seen in the burndown chart, as we all worked together for two full days instead of dividing the work up in several days. This helped our our productablility significantly and meant we overcame obstacles faster.
The project backlog and the graph with the hours we spent are at the end of this document in appendix \ref{sec:Scrum Material}. The reader should note that the graph are vector graphics, so zooming is posible. \\
As can be seen from our backlog we have added some user stories and kept on working on some from previous sprints.
\subsection{User Stories}
For sprint \# 5 we revisited the following user stories: \\
\begin{verbatim}
	As an analytic
	I want to save data for later sessions
	So I can continue working on datasets
\end{verbatim}
\begin{verbatim}
As an analytic
I want to make pie charts, column charts, bar charts, etc
So that I can make data more presentable
\end{verbatim}
\begin{verbatim}
As an analytic
I want the graphics to scale to the screen size, when i turn the tablet between landscape and portrait format
So the graphics are changed to fit the screen
\end{verbatim}
\subsection{Tasks} % (fold)
\label{sub:Tasks}
The stories are divided up into the following tasks
% subsection Tasks (end)Tasks
\begin{itemize}
	\item Implement Dynamic import of data into graphs.
	\item Implement save functionality for graphs.
	\item Create profiling for the graphs to recognize the use of landscape and portrait mode.
\end{itemize}










% section Sprint Material (end)
