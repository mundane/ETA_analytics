\section{Sprint Material} % (fold)
\label{sec:Sprint Material}
\subsection{Version} % (fold)
\label{sub:Version}
The current state of our app has version 0.0.1. Meaning that the app is still alpha quality, with some functionality implementet, but lot of unfinished work.
% subsection Version (end)
\subsection{Source code} % (fold)
\label{sub:Source code}
We still use Git and GitHub. The source code is available at: \verb!https://github.com/mundane/ETA_analytics!, in the app folder. The other folders contains prototypes and experiments.
% subsection Source code (end)
\subsection{Sprint Explanation}
The burndown chart shows that, in spite of exams and easter holidays stalled the progress of development in the beginning of the sprint, we did manage to finish some of the tasks within the sprint time frame.
\begin{figure}[h!]
  \centering
    \includegraphics[width=0.8\textwidth]{images/burndown.png}
	\caption{Burndown chart for sprint \# 2. Easter holidays and exams is the reason for the steep curve.}
\end{figure}

\subsection{User Stories}
For sprint \# 2 we had the following user stories: \\
As an analytic \\
I want to make pie charts, column charts, bar charts, etc \\
So that I can make data more presentable \\

As an analytic \\
I want to make charts interactive  \\
So that I can make data more presentable \\

As an analytic \\
I want a way to show selected data from the chart \\
So that I can make data more presentable\\

As an analytic \\
I want to save data for later sessions \\
So I can continue working on datasets \\

\subsection{Tasks} % (fold)
\label{sub:Tasks}
We divided the stories up in the following tasks
% subsection Tasks (end)Tasks
\begin{itemize}
	\item Read up on the MVC model
	\item Research testing tools for Sencha
	\item Research JSLint
	\item Convert CSV format to JSON
	\item Create datagrid from JSON
	\item Make integration to Jstat in the datagrid view
	\item Create charts
\end{itemize}










% section Sprint Material (end)
