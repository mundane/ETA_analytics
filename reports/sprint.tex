\section{Sprint Material} % (fold)
\label{sec:Sprint Material}
\subsection{Sprint Explanation}
The burn down chart shows that, although exams and vacations stalled the progress of development, we did manage to finish some of the tasks within the Sprint fairly quick  .Some of the tasks includes updates and linking across sprints 1 and 2.

Lack of solid software architecture which was an issue in the previous sprint was solved using a MVC Model,This significantly helped in improving productivity,Unit testing based on this architecture was also implemented.
\subsection{User Stories}
For sprint \# 2 we had the following user stories: \\
As an analytic \\
I want to make pie charts, column charts, bar charts, etc \\
So that I can make data more presentable \\

As an analytic \\
I want to make charts interactive  \\
So that I can make data more presentable \\

As an analytic \\
I want a way to show selected data from the chart \\
So that I can make data more presentable\\

As an analytic \\
I want to save data for later sessions \\
So I can continue working on datasets \\

\subsection{Tasks} % (fold)
\label{sub:Tasks}
We divided the stories up in the following tasks
% subsection Tasks (end)Tasks
\begin{itemize}
	\item Read up on the MVC model
	\item Research testing tools for Sencha
	\item Research JSLint
	\item Convert CSV format to JSON
	\item Create datagrid from JSON
	\item Make integration to Jstat in the datagrid view
	\item Create charts
\end{itemize}










% section Sprint Material (end)
