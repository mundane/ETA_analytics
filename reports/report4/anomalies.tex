\section{Anomalies} % (fold)
\label{sec:Anomalies}

\subsection{Code Inspection} % (fold)
\label{sub:Code Inspection}



CodeCode Inspection workshop involved us, presenting project source code to an inspection team,
who upon reviewing, complied a set of suggestions and best practices.

Since the feedback from the review team was low and not comprehensive, further addition inhouse inspection was conducted on possible 
code refactoring and possible rework.

The comments received are presented below:

Comments on Source code: ``The source code has to be commented and appropriate description fucntionalities should be added''.

Consider JavaScript Compressor: ``As rich applications is built with larger JavaScript code bases, the need for JavaScript compression 
to keep bandwidth and page load times as small as possible is becoming more important for faster load times and more enjoyable user experiences''.

Consider Jquery:Think of using jQuery(might speed up things/less code: jQuery is a cross-browser JavaScript library designed to simplify the client-side scripting of HTML
 
Consider Jslint: is a web-app which takes a JavaScript source and scans it. If it finds a problem, it returns a message describing the problem and an approximate solution. 

% subsection Code Inspection (end)
\subsection{Rework and Code Refactoring} % (fold)
\label{sub:rework}

% subsection subsection name (end)

Code Documentation guide:Comments on our source codes and fucntionalities description has been added using Docco, see section \ref{sec:Documentation}

JavaScript Compressor: A statergy to make use of a JavaScript optimizer later in the project finsihing stages has been decided.
Closure Compiler- a google service for java code optimizer will be used for this purpose.Instead of compiling from a source language to machine code, it compiles from JavaScript to better JavaScript. It parses your JavaScript, analyzes it, removes dead code and rewrites and minimizes what's left. 
It also checks syntax, variable references, and types, and warns about common JavaScript pitfalls. 

Jquery: We have decided not to use Jquery instead of Sencha because, Sencha is far more extensive than its competitors, with a vast array of UI    
components, explicit iOS and Android support, storage and data binding facilities using JSON and HTML5 offline storage, and more.

For all that apparent extra weight and bulky library, Sencha performed better and was more reliable.

% section Anomalies (end)