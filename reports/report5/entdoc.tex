\section{Documentation} % (fold)
\label{sec:Documentation}
Chosing to build our application in JavaScript means that finding a documenting tool was not particular easy, as tools like JavaDoc and doxygen are not applicable to JavaScript. In our research we happened upon JSdoc, \url{http://code.google.com/p/jsdoc-toolkit}, dox, \url{https://github.com/visionmedia/dox}, and docco, \url{http://jashkenas.github.com/docco/}. JSdoc had the advantage that they use the same anotation as JavaDoc and also produced output that is visually and structurally similar to what JavaDoc outputs. The main downside of JSdoc is the setup. The cli version requires a rather cumbersome configuration to work with a project like ours, in that we rely heavily on other frameworks, code that we do not need to be handled by the documentation tool. Considering the downsides of JSdoc we looked for a more light weight tool and found dox, which is very light weight and supports a rather large subset of the same anotations as JSdoc. Unfortunately it only supports JSON output, so if we were to use it we would have to write our own templates to render the JSON in human readable form. We did find a project made specifically for dox, which had a template but it was broken. The final tool we found, docco, is different from the others in its approach to document code. In stead of anotating method, functions, fields and class etc. Documentation made with docco puts the descriptions written by the developer, right next to the code. This can help while reading and understading the code. 
% section Entity Documentation (end)