\section{Process improvements} % (fold)
\label{sec:improvements}
To do process improvements, we had to investigate our existing processes. We looked at our earlier sprint retrospectives, to see what issues we had experienced before.
\begin{itemize}
	\item We underestimated the time needed for many tasks, often because we had no prior experience with the subject and therefore had to research more than anticipated, this had the consequence, that we have not been able to implement as much as we hoped for, when we started the project.
	\item In the first sprints, we often neglected to update the sprint material in time, which meant the information wasn't accurate. This cost time in the end of the sprints, as we had to fill out the missing data.
\end{itemize}

We had a workshop in the group, where we discussed what issues we saw in our project. One major issue was, that we didn't believe we could implement all that was planned in the first sprint.
we therefore had a meeting with the product owner, where we talked about downsizing the project, because of the time issues. This was agreed upon, so we now only focus on importing a CSV file, showing the content of the file, and producing graphs of the data.

Another issue that came up in the workshop, was the quality of our code. We would like to refactor it to make it more efficient, and to comprimize it in the end to make it run faster.
\subsection{Process} % (fold)
\label{sec:Process}

Our software development project has been well defined through out and in order to check if it is capable of producing products that meet our product owner's needs we decided to do a reflective process analysis of all the previous sprints.

The objectives of this process improvement in our case is to:

\begin{itemize}
	\item Understand the qualities of our current software development processe and the factors that affect process capability.
	\item Plan and implement actions that will modify the processes so as to have a better meet developmental process.
	\item Assess the impacts and benefits gained.
\end{itemize}

\subsection{Defining Processes:}

Reflecting back at the sprint we notice that:

\begin{itemize}
	\item Tasks and the responsibilities of the individuals have been defined tightly, to ensure everyone involved in the process was aware  of their jobs and responsibilties.
	\item A well defined user stories through out the represent task at each sprint.
	\item Development platform and technologies review and chosen precisely.
	\item Defined standards for product functionalities.
	\item Feedback on product from all perspective and from other team members has been logged.
\end{itemize}

We find that the processes were defined based on the supported platforms and used accordingly to mere technical objectives. Initially the process were defined on the basis to identify existing technologies,abilities to ensure performance of the processes. For example: Timely meeting among product owner and team members, agreement on technologies to be used, research materials, code standards etc.

Further the infrastructures were carefully selected to meet our requirement.
Eg: Github for code repository and Google Docs for SCRUM management and the target devices we chosen on prior experience and knowledge of the core functionalities within.

Ensured that the members had the right skills and training to meet the user stories so that one has had the ability to execute and sustain the processes.

\subsection{Process measurement.}

A detailed analysis of the work progress was made based on the data collected from our source code repository and SCRUM tool.

\textbf{Task vs Time management.}

We find that there was a nice balance in the hour's logged by the team over the week. On taking an average of the task completed by the team over the week the following trends was observed.
\begin{figure}[!ht]
  \centering
    \includegraphics[width=1.0\textwidth]{images/commits.png}
  \caption{Commits by day. Larger circles indicate more commits.}
\end{figure}

% section Process Improvements (end)
\subsubsection{Synchronized source code repository.} % (fold)
\label{ssub:Synchronized source code repository.}

Our Git repository was managed by a single developer(Asger), to audit the legibility of the commits and upon review commits were made by the Git master to keep the repository free from conflicts.
% subsubsection Synchronized source code repository. (end)
\begin{figure}[!ht]
  \centering
    \includegraphics[width=1.0\textwidth]{images/contrib.png}
  \caption{Commits by contributors.}
\end{figure}

Based on the measurements we found we conducted workshop to refelct on the process and identify the key issues. We have  some areas of concern however there were plenty of things that went right too.

\begin{itemize}
	\item	Overall software architecture was insightful and added flexibility to make improvements. Implementation of core functionalities of  the system.
	\item	Synchronized code repository.
	\item	Product testing based on static analysis tool JSLint and Siesta was surprisingly easy.
	\item	A plug-in free pure JavaScript based File Browser.
	\item	Better Sprint Delegation:Better coordination on task and work assignment.
	\item	Refactoring Code:Updated Git with re-factored code.
	\item	Decided on and set up documentation tool for JavaScript.
\end{itemize}

However few tasks some of the processes that needs more attentions were:
\begin{itemize}
	\item Members did not always update the sprint backlog.
	\item We underestimated some tasks and therefore used more time.
	\item Time spent on researching tool and methods.
	\item Consistency in code integration.
	\item We had few physical meetings, as many members are busy with other courses and work.
\end{itemize}

\subsection{Process Change}

